\documentclass[a4paper,12pt]{article}
\usepackage[T2A]{fontenc}
\usepackage[utf8]{inputenc}
\usepackage[english,russian]{babel}
\usepackage{amsmath,amsfonts,amssymb,amsthm,mathtools}
\usepackage{tikz}
\author{Баширов 778}
\title{Домашнее задание 2}

\begin{document}
\maketitle
\newpage

\large\textbf{1.1}\normalsize\\
колличество всех возможных вариантов: $2^{10}$\\
колличество вариантов удовл заданию: $C_{10}^5$\\
ответ: $\frac{C_{10}^5}{2^{10}}$\\

\large\textbf{1.2}\normalsize\\
из первого пункта\\
$\frac{1 - \frac{C_{10}^5}{2^{10}}}{2}$\\

\large\textbf{1.3}\normalsize\\
$1/2$ так как условиям удовл 2 из 4 исходов $i$го и $(11-i)$го бросков\\

\large\textbf{2}\normalsize\\
по формуле Байеса:\\
$P(6|"6") = \frac{P("6"\mid6)*P(6)}{P("6")} = \frac{0,75*(1/6)}{0,75*(1/6)+(5/6)*(1/20)}$\\

\large\textbf{3}\normalsize\\
$E(max{X_{1},X_{2}}+min{X_{1},X_{2}}) = E(X_{1}+X_{2}) = E(X_{1}+X_{2}) = 2*E(X) = 7$\\

\large\textbf{4.1}\normalsize\\
Запишем рекурсивное уравнение:\\
$E = (1/6)*(1/6)*2 + (1/6)*(5/6)(E + 2) + (5/6)*(E + 1)$\\
Получим:\\
$E = 42 $\\

\large\textbf{5.1}\normalsize\\
Запишем рекурсивное уравнение:\\
$P(x) = P(x-1)*p + P(x+1)*(1-p)$\\
$P(0) = 1$\\
$P(10) = 0$\\
Получим:\\
$P(5)=0,03$\\

\large\textbf{6}\normalsize\\
$P(AB) = P(A)*P(B) = 1/6$
\end{document}