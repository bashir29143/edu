\documentclass[a4paper,12pt]{article}
\usepackage[T2A]{fontenc}
\usepackage[utf8]{inputenc}
\usepackage[english,russian]{babel}
\usepackage{amsmath,amsfonts,amssymb,amsthm,mathtools}
\usepackage{tikz}
\author{Баширов 778}
\title{Домашнее задание 2}

\begin{document}
\maketitle
\newpage

\large\textbf{2}\normalsize\\
Сперва докажем, что если функция выполнима то существует протыкающее множество. Так как КНФ выполнима, существует такой набор переменных на котором она принимает значение $1$. Зафиксируем набор. Тогда в протыкающее множество будет входить $x_{i}$ если на этом наборе $x_{i} = 1$ или $\neg x_{i}$, если $x_{i} = 1$. Докажем от обратного что данное множество протыкающее. В наше множество точно входит либо $x_{i}$ либо $\neg x_{i}$, а значит базовое множество и множества $A_{i}$ протыкаются. Тогда пусть существует такое множество $A_{C}$, что оно не имеет общих элементов с нашим множеством. Тогда дизъюнкт соответствующий этому набору будет принимать значение $0$ на зафиксированном наборе, а значит и функция примет $0$. Противоречие.\\
Теперь докажем в другую сторону. Пусть есть некоторое протыкающее множество. Тогда пусть все переменные этого множества принимают единицу, функция принимает значение $1$ на данном наборе, так как в каждом дизъюнкте есть пересечение с протыкающим множеством. 


\end{document}