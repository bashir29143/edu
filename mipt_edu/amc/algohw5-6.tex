\documentclass[a4paper,12pt]{article}
\usepackage[T2A]{fontenc}
\usepackage[utf8]{inputenc}
\usepackage[english,russian]{babel}
\usepackage{amsmath,amsfonts,amssymb,amsthm,mathtools}
\usepackage{tikz}
\author{Баширов 778}
\title{Домашнее задание 5-6}

\begin{document}
\maketitle
\newpage
\noindent \large\textbf{1}\normalsize\\
По определению PSPACE-hard:\\
$\forall L \in PSPACE$\\ $L\leq_{p} A \leq_{p} B \Rightarrow L \leq_{p} B$\\
\large\textbf{2}\normalsize\\
По определению PSPACE-hard:\\
$\forall L \in PSPACE$\\ $L\leq_{p} A \in P \Rightarrow L \in P$\\$PSPACE \subseteq P$ \\Из семинара: $P \subseteq PSPACE$ \\Получаем: $P = PSPACE$\\
\large\textbf{3}\normalsize\\
По определению PSPACE-hard:\\
$\forall L \in PSPACE$\\ $L\leq_{p} A \in NP \Rightarrow L \in NP$\\$PSPACE \subseteq NP$ \\Из семинара: $NP \subseteq PSPACE$ \\Получаем: $NP = PSPACE$\\
\large\textbf{4}\normalsize\\
$L \in BPP$\\
$\forall x \in L$ ВМТ выдает $1$ с вероятностью $p_1 > 2/3$\\ $\forall y \notin L$ ВМТ выдает $1$ с вероятностью $p_2 > 2/3$\\ 
Рассмотрим дополнение языка $L$ и поменям выходы $0$ и $1$ на исходной ВМТ. Тогда:\\на $y$ ВМТ выдает $1$ с вероятностью $p_2 > 2/3$\\  на $x$ ВМТ выдает $1$ с вероятностью $p_1 > 2/3$\\Получаем что $\overline{L} \in BPP$\\
\large\textbf{5}\normalsize\\
$L \in RP$\\
$\forall x \in L \exists r : V(x,r)$ выдает $1$\\ Примем $r$ за сертификат\\ Так как в худшем случае V(x,r) работает за полином то $L \in NP$\\
\large\textbf{6}\normalsize\\
\large\textbf{7}\normalsize\\
Для начала найдем среднее время: \\
Назовем шагом колличество карт под картой номер $(n-1)$. Перемешивание останавливается на шаге $(n-1)$. Введем обозначаение 
$E_{i}t$ -- матожидание времени перемешивания после шага $i$. Очевидно, $E_{n-1}t = 1$.\\ Составим рекуррентное уравнение:\\
$E_{i} = (1-\frac{i+1}{n-1})(E_{i} + 1) + \frac{i+1}{n-1}(E_{i+1} + 1)$, где $\frac{i+1}{n-1}$ -- вероятность что после перестановки верхняя карта окажется под картой номер $n-1$, те шаг увеличится. Искомая величина $E_{1} = 1 + \sum_{i=1}^{n-2} \frac{n-1}{i+1}$\\ 
Худшее время равно бесконечности так как можно менять две верхние карты местами сколь угодно большое время.

\end{document}