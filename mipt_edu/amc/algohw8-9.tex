\documentclass[a4paper,12pt]{article}
\usepackage[T2A]{fontenc}
\usepackage[utf8]{inputenc}
\usepackage[english,russian]{babel}
\usepackage{amsmath,amsfonts,amssymb,amsthm,mathtools}
\usepackage{tikz}
\author{Баширов 778}
\title{Домашнее задание 7}

\begin{document}
\maketitle
\newpage
\noindent \large\textbf{1.1}\normalsize\\
$A = 0*x^3 + 0*x^2 + 2x + 3$\\
$B = 0*x^3 + 1*x^2 + 0*x - 1$\\
Используем рекурсию:\\
$A = x*(0*x^2 + 2) + (0*x^2 + 3)$\\
$B = x*(0*x^2 + 0) + (x^2 - 1)$\\
Посчитаем значения в точках $1, i, -1, -i$ перемножим.\\
Искомый многочлен имеет следующие значения:\\
в точке $1$: $0$ \\
в точке $i$: $-4*i - 6$\\
в точке $-1$: $0$\\
в точке $-i$: $4*i - 6$\\

Произведя обратное преобразование Фурье получаем:\\
$(a_0, a_1, a_2, a_3)^T=(-3, -2, 3, 2)^T$\\
$A*B = 2x^3 + 3x^2 - 2x - 3$\\

\large\textbf{1.2}\normalsize\\
Посчитаем значения по формуле из лекции:\\
$8*a_0 = 10*w_8^0 + (3*2^(1/2)i +2i +2)*w_8^0 + 0*w_8^0 + (3*2^(1/2)i -2i +2)*w_8^0 - 2*w_8^0 + (-3*2^(1/2)i +2i +2)*w_8^0 + 0*w_8^0 - (3*2^(1/2)i +2i -2)*w_8^0$\\
$8*a_1 = 10*w_8^0 + (3*2^(1/2)i +2i +2)*w_8^1 + 0*w_8^2 + (3*2^(1/2)i -2i +2)*w_8^3 - 2*w_8^4 + (-3*2^(1/2)i +2i +2)*w_8^5 + 0*w_8^6 - (3*2^(1/2)i +2i -2)*w_8^7$\\
И так далее\\

\large\textbf{2.1}\normalsize\\
Из семинара мы узнали что требуется посчитать след сумму:\\
$\Sigma_{j=0}^{m-1} p_j^3t_{i+j} -2p_j^2t_{i+j}^2 + p_{j}t_{i+j}^3$\\
По аналогии с семинаром посчитаем суммы как коэф произведения многочленов, только теперь возьмем другие многочлены а именно:\\
$P_2 = p_0^2x^{m-1} + ... + p_{m-1}^2$\\
$P_3 = p_0^3x^{m-1} + ... + p_{m-1}^3$\\
$T_2 = t_0^2x^{n-1} + ... + t_{n-1}^2$\\
$T_3 = t_0^3x^{n-1} + ... + t_{n-1}^3$\\
Произведения многочленов посчитаем с помощью БПФ за O(n*logn) коэф самих многочленов за O(n) и сложим все за O(1).\\
Итого O(n*logn)\\
\large\textbf{2.2}\normalsize\\
\large\textbf{3}\normalsize\\
\large\textbf{4.1}\normalsize\\
\large\textbf{4.2}\normalsize\\
\large\textbf{4.3}\normalsize\\
\large\textbf{4.4}\normalsize\\
\end{document}