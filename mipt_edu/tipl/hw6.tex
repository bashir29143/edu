\documentclass[a4paper,12pt]{article}
\usepackage[T2A]{fontenc}
\usepackage[utf8]{inputenc}
\usepackage[english,russian]{babel}
\usepackage{amsmath,amsfonts,amssymb,amsthm,mathtools}
\usepackage{tikz}
\author{Баширов 778}
\title{Домашнее задание 6}

\begin{document}
\maketitle
\newpage

\section{}

\section{}
$L = L_{1}\cup R$ \\
$L_{1} = (L\setminus R)\cup (L_{1}\cap R)$\\
$L_{1}\cap R$ -- конечный язык ($R$ -- конечный) а следовательно регулярный\\
$L\setminus R = L\cap \overline{R}$ -- регулярный язык тк отрицание регулярного и пересечение регулярных языков тоже регулярные языки\\
Следовательно $L_{1}$ регулярный язык как объединение регулярных\\
Ответ: нет

\section{}
Предположим что классами эквивалентности будут множества с одинаковыми остатками от деления на три\\
Докажем предположение\\
1.\\
При приписывании нуля справа остаток изменится следующим образом\\
$d_{2} = d_{1}\times 2 (mod 3)$\\
При приписывании единицы справа остаток изменится следующим образом\\
$d_{2} = d_{1}\times 2 + 1(mod 3)$\\
Последовательным приписыванием нулей и единиц справа можно приписать любое слово и принадлежность результата к какому либо классу будет зависеть только от остатка изначального слова те его принадлежности к некоторому классу\\
Так как язык L является классом эквивалентности то все слова из некоторого класса эквивалентности попарно эквивалентны\\
2.\\
Если слова принадлежат разным классам то они не эквивалентны так как у них разные остатки и при приписывании справа одинакового слова остатки результатов конкатенации тоже будут разные\\
3.\\
У любого числа есть остаток\\
Значит объедининение всех классов смежности равно языку всех слов\\
\\
\\
Язык L -- регулярный так как колличество эквивалентных классов конечно
\section{}
\textbf{a}\\
Каждое слово является классом смежности\\
Докажем это утверждение\\
для любого x,y такого что $x\neq y \exists z: xz\in PAL, yz\notin PAL$\\
чтд\\
\textbf{b}\\
Пусть есть три класса эквивалентности: $L_{1}$, $L_{2}$, $L_{3}$\\
$L_{1} = L$\\
$L_{2} = b^*aa*$\\
$L_{3} = b^*$\\
Докажем что это классы эквивалентности:\\
Если в слове есть подслово ab то оно принадлежит $L_{1} = L$ и после приписывания любого слова там останется\\
Если слово заканчивается на a и оно не принадлежит $L_{1} = L$ (принадлежит $L_{2}$) то только при приписывании слова начинающегося на b или содержащего подслова ab результат конкатенации принадлежит $L$\\
В остальных случаях слово будет принадлежать L только после приписывания слова с подсловом ab\\
$L_{2} + L_{3} = b^*a^*$\\
$L_{1} + b^*a^* = \Sigma^*$\\
чтд\\
И так колличество классов конечно\\
Построим автомат:\\
Каждому состоянию соответствует класс смежности\\
Начальное состояние соответствует классу с $\varepsilon$ те классу $L_{3}$\\
Принимающее состояние $L_{1}$\\

\begin{center}
\begin{tikzpicture}[scale=0.2]
\tikzstyle{every node}+=[inner sep=0pt]
\draw [black] (18.9,-26.8) circle (3);
\draw (18.9,-26.8) node {$L3$};
\draw [black] (43.3,-26.8) circle (3);
\draw (43.3,-26.8) node {$L2$};
\draw [black] (43.3,-43.8) circle (3);
\draw (43.3,-43.8) node {$L1$};
\draw [black] (43.3,-43.8) circle (2.4);
\draw [black] (21.9,-26.8) -- (40.3,-26.8);
\fill [black] (40.3,-26.8) -- (39.5,-26.3) -- (39.5,-27.3);
\draw (31.1,-27.3) node [below] {$a$};
\draw [black] (42.299,-23.984) arc (227.29487:-60.70513:2.25);
\draw (44.36,-19.5) node [above] {$a$};
\fill [black] (44.93,-24.29) -- (45.84,-24.04) -- (45.1,-23.37);
\draw [black] (43.3,-29.8) -- (43.3,-40.8);
\fill [black] (43.3,-40.8) -- (43.8,-40) -- (42.8,-40);
\draw (42.8,-35.3) node [left] {$b$};
\draw [black] (45.98,-42.477) arc (144:-144:2.25);
\draw (50.55,-43.8) node [right] {$a,b$};
\fill [black] (45.98,-45.12) -- (46.33,-46) -- (46.92,-45.19);
\draw [black] (20.223,-29.48) arc (54:-234:2.25);
\draw (18.9,-34.05) node [below] {$b$};
\fill [black] (17.58,-29.48) -- (16.7,-29.83) -- (17.51,-30.42);
\draw [black] (20.223,-29.48) arc (54:-234:2.25);
\fill [black] (17.58,-29.48) -- (16.7,-29.83) -- (17.51,-30.42);
\end{tikzpicture}
\end{center}


\section{}
a.\\
Всегда можем взять $y = \varepsilon$\\
Тогда язык равен $\Sigma^*a\Sigma^*$\\
Те язык регулярный
\end{document}