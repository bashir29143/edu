\documentclass[a4paper,12pt]{article}
\usepackage[T2A]{fontenc}
\usepackage[utf8]{inputenc}
\usepackage[english,russian]{babel}
\usepackage{amsmath,amsfonts,amssymb,amsthm,mathtools}
\usepackage{tikz}
\author{Баширов 778}
\title{Домашнее задание 10}

\begin{document}
\maketitle
\newpage

\section{}
\section{}
$\lbrace a^nb^mb^nc^m | n, m \geq 0\rbrace$ \\
Заметим что это тоже самое что\\
$\lbrace a^nb^nb^mc^m | n, m \geq 0\rbrace$ \\
Построим грамматику для данного языка:\\
$S\rightarrow AB $\\
$A\rightarrow aAb$\\
$B\rightarrow bBc$\\
\section{}
1)\\
$A\setminus R = A\cap (\Sigma^*\setminus R) = A\cap R$\\
Языку A соответствует мп-автомат\\
Языку R соответствует ДКА\\
Тогда языку $A\cap R$ соответствует произведение мп-автомата на ДКА,те новый мп-автомат. Получается что $A\setminus R\in CFL$\\
2)\\
Возьмем за A язык из первой задачи $\Sigma^*\setminus {a^nb^nc^n|n\geq 0}$\\
Пусть $R = \Sigma^*$\\
Тогда $R\setminus A = {a^nb^nc^n|n\geq 0}\notin CFL$\\
Противоречие
3)\\
$A^R \in CFL$
Пусть языку A соответствует грамматика G. Тогда если все правые части правил развернуть то новая грамматика порождает язык $A^R$. 
\section{}
Докажем, что язык $L={\omega t\omega^R} \notin CFL$\\
Отрицание леммы о накачке:\\
$\forall p$\\
$\exists \theta \in L : \forall$ разбиения $\theta = xuyvz:$ \\$ |uv|\geq 1 $\\$|uyv|\leq p $\\$ \exists i>0 : xu^iyv^iz \notin L$\\
Возьмем слово $a^pb^pa^p$
Рассмотрим два случая: когда u и v лежат во втором или третьем сегменте целиком, одно из двух подслов лежит на пересечении сегментов.
1)Очевидно
2)Очевидно
чтд
\section{}

\end{document}
