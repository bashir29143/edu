\documentclass[a4paper,12pt]{article}
\usepackage[T2A]{fontenc}
\usepackage[utf8]{inputenc}
\usepackage[english,russian]{babel}
\usepackage{amsmath,amsfonts,amssymb,amsthm,mathtools}
\author{Баширов 778}
\title{Домашнее задание 1}

\begin{document}
\maketitle
\newpage
\section{}

\textbf{1}

$(b,1)\notin\{1,2,3\}\times\{a,b\}$

Так как $b\notin\{1,2,3\}$

\textbf{2}
$\vert A \times B \vert = \vert A \vert \vert B \vert$

Так как элемент из $A \times B$ представляет собой упорядоченную паруб один из элементов которой является элементом первого множества, а второй -- второго.
Таких пар $\vert A \vert \vert B \vert$ штук.

\textbf{3}

По определению $A \times B = \{ (a,b) \vert a \in A,b \in B \}$

Но $\nexists b : b \in \varnothing$

Следовательно $N \times \varnothing = \varnothing$

\section{}

\textbf{1}

Переберем все варианты

5 -- 1 ababa

4 -- 2 abab; baba

3 -- 2 aba; bab

2 -- 2 ab; ba

1 -- 2 a; b

0 -- 1 
Ответ: 10

\textbf{2}

a)5 б)3 в)2
г) 6 -- кол-во мест в которые можно вставить пустое слово между буквами

\textbf{3}

Нет

пустое слово нельзя

\section{}

$A = \{a,a^3,a^5...\} = \{a^n\vert n\equiv1(mod 2), n > 0\}$

$A \bullet A = \{a^(n+m) \vert n+m=0(mod 2), n,m > 1 \}$ 

\section{} 

а) $(a \vert b)^*a(a \vert b)^*b(a \vert b)^* + (a \vert b)^*b(a \vert b)^*a(a \vert b)^*$

б) $(a)^*ab(b)^*(a)^*$

в) $(b)^*(a)^*$
  
\end{document}
