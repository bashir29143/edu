\documentclass[a4paper,12pt]{article}
\usepackage[T2A]{fontenc}
\usepackage[utf8]{inputenc}
\usepackage[english,russian]{babel}
\usepackage{amsmath,amsfonts,amssymb,amsthm,mathtools}
\usepackage{tikz}
\author{Баширов 778}
\title{Домашнее задание 7}

\begin{document}
\maketitle
\newpage

\section{}
Опишем алгоритм построения праволинейной грамматики по автомату:\\
1) каждому состоянию будет соответствовать одноименный нетерминал(например $q_1$ и $Q_1$)\\
2) на нетерминал соответствующий начальному состоянию заменяется аксиома\\
3) любой нетерминал соответствующий принимающему состоянию заменяется на пустое слово\\
4) каждому $\gamma$-переходу из A в B (где $\gamma$ -- буква или пустое слово) соответствует правило $A \rightarrow \gamma B$\\
Теперь докажем корректность этого алгоритма:\\
1) Докажем что язык принимаемый автоматом вложен в язык порождаемый грамматикой
Произвольное слово принимаемое автоматом можно представить как последовательность состояний автомата с переходами между ними. Заменим состояния на соответствующие им нетерминалы а переходы(если они существуют) на соответствующие правила. Получим последовательность правил начинающуюся с нетерминала соответствующему начальному состоянию. Добавим в начало последовательности правило замены аксиомы на нетерминал соответствующий начальному состоянию а в конец правило замены нетерминала соответствующего принимающему состоянию на пустое слово. В итоге получим последовательность правил результатом которой будет изначальное слово.\\
2) Доказательство что язык порождаемый грамматикой вложен в язык принимаемый автоматом аналогично предыдущему пункту (замена последовательности правил на последовательность переходов).
чтд\\
Теперь построим по данному автомату грамматику:\\
$S\rightarrow Q_0$\\
$Q_0\rightarrow Q_1$\\
$Q_0\rightarrow aQ_3$\\
$Q_1\rightarrow aQ_2$\\
$Q_1\rightarrow bQ_3$\\
$Q_2\rightarrow aQ_3$\\
$Q_2\rightarrow \varepsilon$\\
$Q_3\rightarrow bQ_4$\\
$Q_4\rightarrow Q_0$\\
$Q_4\rightarrow \varepsilon$\\
$Q_4\rightarrow aQ_0$\\

\section{}
Автомат $\cal{A}$:
\begin{center}
\begin{tikzpicture}[scale=0.2]
\tikzstyle{every node}+=[inner sep=0pt]
\draw [black] (15.9,-42.8) circle (3);
\draw (15.9,-42.8) node {$S$};
\draw [black] (15.9,-42.8) circle (2.4);
\draw [black] (15.7,-35.5) circle (3);
\draw (15.7,-35.5) node {$Q_2$};
\draw [black] (15.9,-28.3) circle (3);
\draw (15.9,-28.3) node {$Q_3$};
\draw [black] (15.9,-20.3) circle (3);
\draw (15.9,-20.3) node {$A$};
\draw [black] (24.9,-14.3) circle (3);
\draw (24.9,-14.3) node {$Q_4$};
\draw [black] (33.8,-7.9) circle (3);
\draw (33.8,-7.9) node {$Q_5$};
\draw [black] (33.8,-7.9) circle (2.4);
\draw [black] (29.5,-42.8) circle (3);
\draw (29.5,-42.8) node {$Q_1$};
\draw [black] (48.1,-42.8) circle (3);
\draw (48.1,-42.8) node {$B$};
\draw [black] (10.1,-42.8) -- (12.9,-42.8);
\fill [black] (12.9,-42.8) -- (12.1,-42.3) -- (12.1,-43.3);
\draw [black] (15.9,-25.3) -- (15.9,-23.3);
\fill [black] (15.9,-23.3) -- (15.4,-24.1) -- (16.4,-24.1);
\draw (16.4,-24.3) node [right] {$a$};
\draw [black] (15.78,-32.5) -- (15.82,-31.3);
\fill [black] (15.82,-31.3) -- (15.29,-32.08) -- (16.29,-32.11);
\draw (16.34,-31.91) node [right] {$b$};
\draw [black] (15.82,-39.8) -- (15.78,-38.5);
\fill [black] (15.78,-38.5) -- (15.3,-39.31) -- (16.3,-39.28);
\draw (16.34,-39.14) node [right] {$a$};
\draw [black] (18.4,-18.64) -- (22.4,-15.96);
\fill [black] (22.4,-15.96) -- (21.46,-15.99) -- (22.02,-16.82);
\draw (21.5,-17.8) node [below] {$a$};
\draw [black] (27.34,-12.55) -- (31.36,-9.65);
\fill [black] (31.36,-9.65) -- (30.42,-9.71) -- (31.01,-10.52);
\draw (30.45,-11.6) node [below] {$a$};
\draw [black] (18.9,-42.8) -- (26.5,-42.8);
\fill [black] (26.5,-42.8) -- (25.7,-42.3) -- (25.7,-43.3);
\draw (22.7,-42.3) node [above] {$a$};
\draw [black] (32.5,-42.8) -- (45.1,-42.8);
\fill [black] (45.1,-42.8) -- (44.3,-42.3) -- (44.3,-43.3);
\draw (38.8,-42.3) node [above] {$b$};
\draw [black] (46.565,-45.373) arc (-35.62214:-144.37786:17.917);
\fill [black] (17.44,-45.37) -- (17.49,-46.31) -- (18.31,-45.73);
\draw (32,-53.35) node [below] {$a$};
\draw [black] (45.349,-41.604) arc (-114.43989:-135.44855:91.607);
\fill [black] (45.35,-41.6) -- (44.83,-40.82) -- (44.41,-41.73);
\draw (29.68,-33.79) node [below] {$a$};
\draw [black] (18.791,-21.101) arc (72.94623:37.16532:54.73);
\fill [black] (18.79,-21.1) -- (19.41,-21.81) -- (19.7,-20.86);
\draw (35.24,-28.06) node [above] {$b$};
\end{tikzpicture}
\end{center}

Построим по данному автомату праволинейную грамматику по алгоритму предложенному в задаче 1\\
$S\rightarrow aQ_2$\\
$S\rightarrow \varepsilon$\\
$S\rightarrow aQ_1$\\
$Q_1\rightarrow bB$\\
$B\rightarrow aS$\\
$B\rightarrow bA$\\
$Q_2\rightarrow bQ_3$\\
$Q_3\rightarrow aA$\\
$A\rightarrow aB$\\
$A\rightarrow aQ_4$\\
$Q_4\rightarrow Q_5$\\
$Q_5\rightarrow \varepsilon$\\
Подставив правила с $Q_k$ слева где $k = 1, 2, 3, 4, 5$ в остальные правила получим эквивалентную грамматику:
$S\rightarrow \varepsilon$\\
$S\rightarrow abB$\\
$B\rightarrow aS$\\
$B\rightarrow bA$\\
$S\rightarrow abaA$\\
$A\rightarrow aB$\\
$A\rightarrow aa$\\
Грамматика равна грамматике из условия задачи. Значит автомат построен правильно (корректность алгоритма из первой задачи)
\section{}
Нет, слово abaaa можно получить двумя разными выводами:\\
1)\\
$S\rightarrow abaA$\\
$A\rightarrow aa$\\
2)\\
$S\rightarrow abaA$\\
$A\rightarrow aB$\\
$B\rightarrow aS$\\
$S\rightarrow \varepsilon$\\
\section{}

\section{}
а)\\
$S\rightarrow aSa$\\
$S\rightarrow bSb$\\
$S\rightarrow a$\\
$S\rightarrow b$\\
$S\rightarrow \varepsilon$\\
1)\\
Вложенность языка порождаемого описанной грамматикой(X) в язык из условия(Y)\\
Все слова из языка X палиндромы, так как слова a, b, $\varepsilon$ палиндромы и слова a$\omega$a, b$\omega$b тоже при условии что $\omega$ палиндром\\ 
2)\\
Вложенность Y в X
Покажем что произвольный полиндром можно вывести с помощью правил грамматики описанной выше. 
Возьмем половину палиндрома(в случае нечетной длины половину от слова которое получится исключением центральной буквы). При считывании a применяем правило 1, b правило 2. После считывания половины палиндрома применяем правило 3, 4, 5 в зависимости от буквы которую мы исключили(если не исключили то пустое слово.\\
б)\\
$S\rightarrow aSBb|\varepsilon|a|b$\\
$B\rightarrow b|\varepsilon$\\
в)\\

\end{document}