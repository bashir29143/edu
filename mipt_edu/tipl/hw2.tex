\documentclass[a4paper,12pt]{article}
\usepackage[T2A]{fontenc}
\usepackage[utf8]{inputenc}
\usepackage[english,russian]{babel}
\usepackage{amsmath,amsfonts,amssymb,amsthm,mathtools}
\usepackage{tikz}
\usetikzlibrary{arrows,automata}
\author{Баширов 778}
\title{Домашнее задание 2}

\begin{document}

\maketitle
\newpage
\section{}

\textbf{1}

$ \varepsilon, b, bb \in L $

$ L \subseteq T$

чтобы это доказать достаточно доказать что в L нет слова с тремя b подряд

1. в словах $\varepsilon$, b, bb не более 2 b подряд

2. пусть в слове x не более 2 b подряд

3. если  в слове x не более 2 b подряд то в словах ax, bax, bbax тоже

$ T \subseteq L$

требуется доказать что L содержит все слова в которых не более 2 b подряд

алгоритм которым можно задать слово в котором не более 2 b подряд

если слово заканчивается на a, то $x_0 = \varepsilon$

1)далее применяем правило $x_n = ax_{n-1}$ пока не наткнемся на подслово aba или bba и применяем соответственно bax или bbax

далее повторяем

2) если слово заканчивается на ab или bb то $x_0 = b$ или $x_0 = bb$

далее проделываем пункт 1 

$L = T$ 

\textbf{2}

$\{a, ba, bba\}^*\{\varepsilon, b, bb\}$

тк $\{\varepsilon, b, bb\}$ слова из пункта 1 правил задания языка, а $\{a, ba, bba\}^*$ любое колличество раз примененное второе правило то $\{a, ba, bba\}^*\{varepsilon, b, bb\} = L = T$

\textbf{3}

\begin{center}
\begin{tikzpicture}[>=stealth',shorten >=1pt,auto,node distance=2cm]
\node[initial, state, accepting] (q_1) {$q_1$};
\node[state, accepting] (q_2) [right of=q_1] {$q_2$};
\node[state, accepting] (q_3) [below of=q_2] {$q_3$};
\node[state] (q_4) [below of=q_3] {$q_4$};

\path[->] (q_1) edge [loop above] node {a} (q_1)
edge [bend left] node {b} (q_2)
(q_2) edge node {b} (q_3)
edge [bend left] node {a} (q_1)
(q_3) edge [bend left] node {b} (q_4)
edge node {a} (q_1)
(q_4) edge [loop above] node {a, b} (q_4);

\end{tikzpicture}
\end{center}

при считывании b из состояния $q_n$ в состояние $q_(n+1)$ при $n+1 < 5$

при считывании a и $n < 4$ в состояние $q_1$

в остальных случаях сохраняется положение

следовательно при считывании более 2 b подряд автомат переводится в непринимающее состояние и остается там

в остальных случаях в принимающем

\section{}

а)

\begin{center}
\begin{tikzpicture}[>=stealth',shorten >=1pt,auto,node distance=2cm]
\node[initial, state, accepting] (q_1) {$q_1$};
\node[state] (q_2) [right of=q_1] {$q_2$};

\path[->] (q_1) edge [loop above] node {1} (q_1)
edge [bend left] node {0} (q_2)
(q_2) edge [loop above] node {1} (q_1)
edge [bend left] node {0} (q_1);


\end{tikzpicture}
\end{center}

б)

\begin{center}
\begin{tikzpicture}[>=stealth',shorten >=1pt,auto,node distance=2cm]
\node[initial, state] (q_1) {$q_1$};
\node[state, accepting] (q_2) [right of=q_1] {$q_2$};

\path[->] (q_1) edge [loop above] node {0} (q_1)
edge [bend left] node {1} (q_2)
(q_2) edge [loop above] node {0} (q_1)
edge [bend left] node {1} (q_1);

\end{tikzpicture}
\end{center}


\end{document}