\documentclass{article}
\usepackage[utf8]{inputenc}
\usepackage[fleqn]{amsmath}
\usepackage{amssymb}
\usepackage{mathtools}

\usepackage[T1,T2A]{fontenc}
\usepackage[utf8]{inputenc}
\usepackage{natbib}
\usepackage{graphicx}
\usepackage[left=2cm,right=2cm,
    top=2cm,bottom=2cm,bindingoffset=0cm]{geometry}
\newcommand\tab[1][0.7cm]{\hspace*{#1}}
\title{Домашняя работа 2 (дедлайн -- 15:00 3.10.19)}

\begin{document}

\maketitle
\textbf{Задача 1} (3 балла)

Пусть $X_1, \dots, X_n$ -- независимые одинаково распределенные случайные 
величины (н.о.р.с.в.) с функцией распределения $F(x)$.
Найти функции распределения $\max\limits_{1 \leq i \leq n} X_i$ и $\min\limits_{1 \leq i 
\leq n} X_i$.\\

\textbf{Решение}:\\

$P(X_{i} < x) = F(x)$\\

$F_{min} = P(\min\limits_{1 \leq i \leq n} X_i < x) = P((X_1 < x) + \dots + (X_n < x)) = 1 - P((X_1 \geq x)*\dots *(X_n \geq x)) =$

$= 1 - \sqcap_{i = 1}^{n} P(X_i \geq x) = 1 - (1-F(x))^n$\\

$F_{max} = P(\max\limits_{1 \leq i \leq n} X_i < x) = P((X_1 < x)*\dots *(X_n < x)) = F^n(x)$\\


\textbf{Задача 2} (3 балла)
Случайная величина $\xi$ имеет функцию распределения $F(x)$. Найти функцию распределения случайной величины $\frac{1}{2}(\xi + \vert \xi \vert)$\\

\textbf{Решение}:\\

$\frac{1}{2}(\xi + \vert \xi \vert) = \eta$\\

$(\xi + \vert \xi \vert) \geq 0$\\

$(\xi + \vert \xi \vert) = 0,$ при $\xi \leq 0$

$(\xi + \vert \xi \vert) = 2*\xi,$ при $\xi > 0$\\

$P(\xi < x) = F(x)$

$P(\eta < x) = F_{\eta}(x)$

$P(\eta < 0) = 0 = F_{\eta}(0)$\\

При $x > 0$: $P(\eta < x) = P(0 < \xi < x) + P(\xi < 0)$

$P(0 < \xi < x) = F(x) - F(0)$

$P(\eta < x) = F(x) - F(0) + F(0)$\\

$F_{\eta}(x) = F(x)$, при $x > 0$

$F_{\eta}(x) = 0$, при $x \leq 0$\\

\textbf{Задача 3} (3 балла)

В круглой комнате произвольным образом провели диаметр и в одном из концов этого диаметра поставили прожектор так, чтобы он мог светить внутрь комнаты (направление, в котором светит прожектор задаётся углом $\alpha$, который отсчитывается от направления проведённого диаметра; $\alpha \in [-\frac{\pi}{2}, \frac{\pi}{2}]$ -- отрицательный угол отсчитывается  по часовой стрелке от направления диаметра, положительный -- против часовой). Найти функцию распределения длины луча от прожектора до стены, если угол $\alpha$ -- это равномерно распределённая случайная величина на отрезке $[-\frac{\pi}{2}, \frac{\pi}{2}]$.\\

\textbf{Решение}:\\

Из геометрии:

$l = 2r*cos\alpha$

$F_l(x) = P(l<x) = P(cos\alpha < \frac{x}{2r}) = 2*P(\alpha > arccos\frac{x}{2r}) = (\frac{\frac{\pi}{2} - arccos\frac{x}{2r}}{\frac{\pi}{2}})$\\

\textbf{Задача 4} (4 балла)

Допустим, что вероятность столкновения молекулы с другими молекулами в
промежутке времени $[t, t + \Delta t)$ равна $p = \lambda\Delta t+o(\Delta t)$ и не зависит от времени, прошедшего
после предыдущего столкновения $(\lambda = const)$. Найти распределение времени свободного
пробега молекулы (показательное распределение) и вероятность того, что это время превысит заданную величину $t^{\star}$\\

\textbf{Решение}:\\

$t$ - время свободного пробега

$p(x) = \lambda x+o(x)$

$F(x) = P(t<x) = p(x)$

$\rho(x) = F'(x) = \lambda + o'(x)$

Доп условие на $o(x)$:

$\int\limits^{+ \infty}_0(\lambda + o'(x))dx = 1$ 

$F'(x) = \frac{F(x+dx) - F(x)}{dx} = \frac{P(t < x+dx) - P(t < x)}{dx} = \frac{P(x < t < x+dx)}{dx} = \frac{(1-p(x))p(x)}{dx} =$

$= \frac{(1 - \lambda x - o(x))(\lambda dx + o(dx))}{dx} = \lambda(1-\lambda x - o(x))$

$\lambda(1-\lambda x - o(x)) = \lambda + o'(x)$, $o(x) = y$

$y' + \lambda y = -\lambda^2x$

Учитывая доп условие получаем:

$y = o(x) = -e^{-2x} - \lambda x +1$

$p(x) = 1 - e^{-\lambda x}$

$P(t>t^*) = 1 - p(t^*) = e^{-\lambda t^*}$

$F(x) = p(x) = 1 - e^{-\lambda x}$

$\rho (x) = F'(x) = \lambda e^{-\lambda x}$\\

\textbf{Задача 5} (2 балла)

Диаметр круга измерен приближенно. Считая, что его величина равномерно распределена в отрезке $[a, b]$, найти распределение площади круга, её среднее значение и дисперсию.\\

\textbf{Решение}:\\

$S = \frac{\pi}{4}d^2$

$F_s(x) = P(S < x) = P(d^2 < \frac{4x}{\pi}) = P(d < \sqrt{\frac{4x}{\pi}}) = \lbrace \frac{\sqrt{\frac{4x}{\pi}} - a}{b - a}$, при $ a < \sqrt{\frac{4x}{\pi}} < b$; $1$, при $b<\sqrt{\frac{4x}{\pi}}$; $0$, при $\sqrt{\frac{4x}{\pi}} < a \rbrace$\\

$\rho_s(x) = F'_s(x) = \frac{1}{b-a}\frac{2}{\sqrt{\pi}}\frac{1}{\sqrt{x}}\frac{1}{2}$ при $\frac{\pi a^2}{4} < x < \frac{\pi b^2}{4}$\\

$ES = \int\limits_{\frac{\pi}{4}a^2}^{\frac{\pi}{4}b^2} \rho_s(x)xdx = \frac{1}{\sqrt{\pi}(b-a)}\frac{2}{3}x^{\frac{3}{2}}\vert^{\frac{\pi}{4}b^2}_{\frac{\pi}{4}a^2} = \frac{2}{3}\frac{(\frac{\pi}{4}b^2)^{\frac{3}{2}}-(\frac{\pi}{4}a^2)^{\frac{3}{2}}}{\sqrt{\pi}(b-a)} = \frac{\pi}{12}\frac{b^3-a^3}{b-a}$\\

$ES^2 = \int\limits_{\frac{\pi}{4}a^2}^{\frac{\pi}{4}b^2} \rho_s(x)x^2dx = \frac{2}{5}\frac{(\frac{\pi}{4}b^2)^{\frac{5}{2}}-(\frac{\pi}{4}a^2)^{\frac{5}{2}}}{\sqrt{\pi}(b-a)} = \frac{\pi^2}{80}\frac{b^5-a^5}{b-a}$\\

$DS = ES^2 - (ES)^2$ \\

\textbf{Задача 6} (2 балла)

Пусть случайные величины $\xi$  и $\eta$ независимы и $\mathbb{E}\xi = 1$, $\mathbb{E}\eta = 2$, $\mathbb{D}\xi = 1$,  $\mathbb{D}\eta = 4$. Найти математические ожидания случайных величин:

а) $\xi^2 + 2\eta^2 - \xi\eta - 4\xi + \eta + 4$; б) $(\xi + \eta + 1)^2$\\


\textbf{Решение}:\\

$D\xi = E\xi^2 - (E\xi)^2$

$E\xi^2 = D\xi + (E\xi)^2 = 2$

$E\eta^2 = 8$\\

a) $E(\xi^2 + 2\eta^2 - \xi\eta - 4\xi + \eta + 4) = E\xi^2 + 2E\eta^2 - E\xi E\eta - 4E\xi + E\eta + 4 = 18$\\

б) $E(\xi + \eta + 1)^2 = E(\xi^2 + 2\xi\eta + \eta^2 + 2\xi + 2\eta +1) = E\xi^2 + 2E\xi E\eta + E\eta^2 + 2E\xi + 2E\eta +1 = 21$\\

\textbf{Задача 7} (3 балла)

\textbf{а)} Пусть $\xi$ -- положительная невырожденная случайная величина с конечным математическим ожиданием. Доказать, что $$\frac{1}{\mathbb{E}\xi} \leq \mathbb{E}\frac{1}{\xi}$$\\

\textbf{Решение}:\\

Так как матожидание положительной случайной величины положительно: 

$\frac{1}{\mathbb{E}\xi} \leq \mathbb{E}\frac{1}{\xi}$ эквивалентно $1\leq \mathbb{E}\frac{1}{\xi}\mathbb{E}\xi$\\

Пусть $\eta_1 = \sqrt{\xi}$ и $\eta_2 = \frac{1}{\sqrt{\xi}}$, тогда из неравенства Коши-Буняковского-Шварца:

$$\mathbb{E}\eta_1^2\eta_2^2 \leq \mathbb{E}\eta_1^2\mathbb{E}\eta_2^2$$
$$1 \leq \mathbb{E}\eta_1^2\mathbb{E}\eta_2^2$$
$$1\leq \mathbb{E}\frac{1}{\xi}\mathbb{E}\xi$$\\

\textbf{б)} Пусть $\xi$ и $\eta$ -- независимые положительные случайная величины, с конечным математическим ожиданием. Доказать, что
$\mathbb{E}\Big(\frac{\xi}{\eta}\Big)^r \geq \frac{\mathbb{E}\xi^r}{\mathbb{E}\eta^r}$\\

\textbf{Решение}:\\

Из пункта а, $\frac{\mathbb{E}\xi^r}{\mathbb{E}\eta^r} \leq \mathbb{E}\xi^r\mathbb{E}\frac{1}{\eta^r}$

Из неравенства Коши-Буняковского-Шварца: $$\mathbb{E}\sqrt{\xi^r}^2*\mathbb{E}\frac{1}{\sqrt{\eta^r}}^2 \leq \mathbb{E}\sqrt{\xi^r}^2\frac{1}{\sqrt{\eta^r}}^2$$

$$\mathbb{E}\xi^r\mathbb{E}\frac{1}{\eta^r} \leq \mathbb{E}\Big(\frac{\xi}{\eta}\Big)^r$$
 
$$\mathbb{E}\xi^r\mathbb{E}\frac{1}{\eta^r} \leq \mathbb{E}\frac{1}{\eta^r}\xi^r$$

Получаем: $$ \mathbb{E}\Big(\frac{\xi}{\eta}\Big)^r\geq \mathbb{E}\xi^r\mathbb{E}\frac{1}{\eta^r} \geq \frac{\mathbb{E}\xi^r}{\mathbb{E}\eta^r}$$\\

\textbf{Задача 8} (5 баллов)

На небольшом кластере GPU прямо сейчас очередь на обучение из 30 нейросетей. Единовременно на кластере может обучаться только одна сеть. За каждую нейросеть отвечают разные ML-инженеры. Нейросети имеют разные свойства: 10 из них больших (время их обучения 15 часов) и 20 маленьких (время их обучения 1 час). Пока не наступил момент начала обучения, разработчик переживает и внимательно следит за очередью, бесполезно растрачивая время. Посчитайте математическое ожидание, сколько человеко-часов будет потрачено на переживания разработчиков ровно с текущего момента (кластер освободился и начинает обрабатывать очередь, описанную выше), если задачи в очереди расположены в случайном порядке\\

\textbf{Решение:}\\

Каждому инженеру дадим номер в очереди начиная с конца, начиная с нуля: от 0 до 29. Затрата в человеко-часах  $\xi$ равна сумме произведений номера инженера на колличество часов требуемые на задачу(1 или 15). Событием $\xi_i$ будет являться некоторая упорядоченная расстановка инженеров в очереди, а значением -- $\sum\limits_{i = 0}^{29}i*t_j$, где $i$ номер инженера в очереди,  а $t_j$ колличество часов требуемые на его задачу. Вероятность любой конкретной упорядоченной расстановки равна $\frac{1}{30!}$. Так как вероятность постоянна ее можно вынести за знак суммы выражения для матожидания: $E\xi = \frac{1}{30!}\sum\sum\limits_{i = 0}^{29}i*t_j$, где внешнее суммирование идет по всем расстановкам очереди. Сумму можно посчитать следующим образом: n-й инженер, который стоит на m-м месте встречается в 29! выборках, а значит в сумму входит 29! раз, таких инженеров 20 с маленькой задачей и 10 с большой. Получаем: $\sum\sum\limits_{i = 0}^{29}i*t_j = 29!*20*\sum\limits_{i = 0}^{29}i*1 + 29!*10*\sum\limits_{i = 0}^{29}i*15 = 73950*29!$.\\

Тогда $E\xi = 2465$


\end{document}