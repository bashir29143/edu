\documentclass{article}
\usepackage[utf8]{inputenc}
\usepackage[fleqn]{amsmath}
\usepackage{amssymb}
\usepackage{mathtools}

\usepackage[T1,T2A]{fontenc}
\usepackage[utf8]{inputenc}
\usepackage{natbib}
\usepackage{graphicx}
\usepackage[left=2cm,right=2cm,
    top=2cm,bottom=2cm,bindingoffset=0cm]{geometry}
\newcommand\tab[1][0.7cm]{\hspace*{#1}}
\newcommand\Gammad{\text{Gamma}}
\newcommand\myp{\mathbb P}
\title{Домашняя работа 6}

\begin{document}

\maketitle
\textbf{Решение 1}\\


\textbf{Решение 2}\\
Заметим что интеграл является формулой свертки с точностью до коэф.\\
Пусть
$$\xi_1 \sim \alpha_1 f_1(x)$$
$$\xi_2 \sim \frac{1}{sqrt{\pi}} f_2(x)$$
$$\xi_3 \sim \frac{1}{sqrt{2\pi}} f_3(x)$$
Из свертки:
$$\xi_1 + \xi_2 = \xi_3$$
$$\xi_1 = \xi_3 - \xi_2$$
Как линейную комбинацию нсв распределнных нормально получаем:\\
$$\xi_1 \sim \frac{1}{\sqrt{3\pi} e^{-\frac{x^2}{3}}$$

\textbf{Решение 3}\\

\textbf{Решение 4}\\

\textbf{Решение 5}\\

\textbf{Решение 6}\\

\end{document}
