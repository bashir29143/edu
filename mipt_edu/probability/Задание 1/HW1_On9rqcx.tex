\documentclass{article}
\usepackage[utf8]{inputenc}
\usepackage[fleqn]{amsmath}
\usepackage{amssymb}
\usepackage{mathtools}

\usepackage[T1,T2A]{fontenc}
\usepackage[utf8]{inputenc}
\usepackage{natbib}
\usepackage{graphicx}
\usepackage[left=2cm,right=2cm,
    top=2cm,bottom=2cm,bindingoffset=0cm]{geometry}
\newcommand\tab[1][0.7cm]{\hspace*{#1}}

\newcommand\myp{\mathbb P}

\title{Домашняя работа 1}

\begin{document}

\maketitle
\textbf{Задача 1} (5 баллов)

Отрезок длины $a_1 + a_2$ поделён на две части длины $a_1$ и $a_2$ соответственно. $n$ точек последовательно бросаются на удачу на отрезок. Найти вероятность того, что ровно $m$ из $n$ точек попадут на часть отрезка длины $a_1$.\\

Если считать, что точки попадают равновероятно на каждую точку отрезка, то вероятность попадания на первую часть равна $p_1 = \frac{a_1}{a_1 + a_2}$, а на вторую $p_2 = \frac{a_2}{a_1 + a_2}$.\\
Колличество исходов эксперимента, которые удовлетворяют условию: $C_n^m$. Вероятность каждого такого исхода: $p_1^m*p_2^{n-m}$.\\
Ответ: $C_n^m*(\frac{a_1}{a_1 + a_2})^m*(\frac{a_2}{a_1 + a_2})^{n-m}$.\\

\textbf{Задача 2} (2 балла)

На плоскость нанесены горизонтальные параллельные прямые на одинаковом расстоянии $a$ друг от друга. На плоскость наудачу бросается монета (круг) радиуса $R$ ($R < \frac{a}{2}$). Найти вероятность того, что монета не пересечёт ни одну из прямых.\\

Координату верхней грани монеты будем счтитать от ближайшей верхней линии. Вероятность координаты монеты будет равномерно распределена от $0$ до $a$. Монета пересекает прямую при координате $x \geq a - 2R$.
Тогда ответ: $\frac{a - 2R}{a}$.\\

\textbf{Задача 3} (3 балла)

На шахматную доску случайным образом ставятся два белых короля. Найти вероятность того, что эти два короля будут бить друг друга.\\

Будем считать, что любая упорядоченная пара на доске равновероятна. Посчитаем колличество расположений пар бьющих друг друга королей. Если первый король размещен в угловых клетках, вариантов для второго 3. Если первый король размещен в краевых (не угловых клетках), то вариантов для второго 5. Если первый король расположен в другой клетке, то вариантов для второго 8. Итого колличество пар удовлетворяющих условию: $4*3 + 24*5 + 36*8 = 420$. Всего пар: $64*63$. Тогда ответ: $p = \frac{420}{64*63} = \frac{5}{48}$.\\

\textbf{Задача 4} (1.5 балла)

В $n$ ящиках размещают $2n$ шаров. Найти вероятность того, что ни один ящик не пуст, если шары неразличимы и все различимые размещения имеют равные вероятности.\\

Вариантов размещения шаров, если ящики  могут быть пустыми: $ C_{n+k-1}^{k-1}$ для $n$ шаров и $k$ ящиков.($k-1$ перегородок и $n$ шаров, имеем $n+k-1$ мест и решаем на какие места разместить перегородки).\\
Вариантов размещения шаров, если ящики не могут быть пустыми: $ C_{n-1}^{k-1}$ для $n$ шаров и $k$ ящиков.($k-1$ перегородок и $n$ шаров, имеем $n-1$ мест и решаем на какие места разместить перегородки).\\
Получаем ответ разделив колличество размещений с непустыми ящиками на общее колличество размещений.\\
Ответ: $\frac{C_{2n-1}^{n-1}}{C_{3n-1}^{n-1}}$\\

\textbf{Задача 5} (\textit{продолжение 4}) (1.5 балла)

Найти вероятность того, что заданный ящик содержит ровно $m$ шаров.\\

Так как один ящик заполнен ровно $m$ шарами, остается $2n-m$ шаров и $n-1$ ящик. Используя формулу из прошлой задачи для размещения шаров, если ящики  могут быть пустыми, получаем ответ: $\frac{C_{3n-m-2}^{n-2}}{C_{3n-1}^{n-1}}$.\\

\textbf{Задача 6} (4 балла)

Пусть $A_1, A_2, \dots$ -- последовательность независимых событий. Доказать, что

\begin{center}
    $\myp(\bigcap\limits_{k=1}^{\infty} A_k) = \prod\limits_{k=1}^{\infty}\myp(A_k)$ 
\end{center}




Используем правило Де Моргана, чтобы преобразовать данное равенство.
\begin{center}
	$\bigcap\limits_{k=1}^{\infty} A_k = \overline{\bigcup\limits_{k=1}^{\infty} \overline{A_k}}$
	
	$1 - \myp(\bigcup\limits_{k=1}^{\infty} \overline{A_k}) = \prod\limits_{k=1}^{\infty}\myp(A_k)$
	
	$\myp(\bigcup\limits_{k=1}^{\infty} \overline{A_k}) = 1 - \prod\limits_{k=1}^{\infty}\myp(A_k)$
\end{center}

Чтобы доказать последнее утверждение, построим последовательность множеств каждое из которых вложено в следующее множество. Такой последовательностью будет последовательность $B_k$, где 
\begin{center}
	$B_k = \overline{A_1}\bigcup\overline{A_2}\bigcup\dots\bigcup\overline{A_k}$ 
\end{center}

Заметим, что:
\begin{center}
	$\bigcup\limits_{k=1}^{n} \overline{A_k} = \bigcup\limits_{k=1}^{n} \overline{B_k}$
	
	$\myp(\bigcup\limits_{k=1}^{n} \overline{B_k}) = \myp(\bigcup\limits_{k=1}^{n} \overline{A_k}) = 1 - \prod\limits_{k=1}^{n}\myp(A_k)$
\end{center}

По теореме непрерывности вероятности:
\begin{center}
	$\myp(\bigcup\limits_{k=1}^{\infty} \overline{B_k}) = \myp(\bigcup\limits_{k=1}^{\infty} \overline{A_k}) = 1 - \prod\limits_{k=1}^{\infty}\myp(A_k)$
\end{center}

Ч.Т.Д.\\

\textbf{Задача 7} (3 балла)

В самолете $n$ мест.
Есть $n$ пассажиров, выстроившихся друг за другом в очередь.
Во главе очереди -- <<заяц>> (пассажир без билета).
У всех, кроме <<зайца>>, есть билет, на котором указан номер посадочного билета.
Так как <<заяц>> входит первым, он случайным образом занимает некоторое 
место.
Каждый следующий пассажир, входящий в салон самолета, действует по такому 
принципу: если его место свободно, то садится на него, если занято, то занимает с 
равной вероятностью любое свободное.
Найдите вероятность того, что последний пассажир сядет на свое место.\\

Составим уравнение для искомой вероятности. Если <<заяц>> садится на единственное не купленное никем место, то последниий пассажир садится на свое место. Если <<заяц>> садится на место i-го в очереди пассажира, где $i \leq n-2$, то задача сводится к изначальной только с колличеством пассажиров. И, наконец, если заяц садится на место одного из двух последних пассажиров, то последний садится не на свое место. Получаем уравнение и граничное условие при очереди из двух пассажиров:

\begin{center}
    $P(n) = \frac{1}{n} + \frac{1}{n}*P(n-1) + \dots + \frac{1}{n}*P(n-i) + \dots + \frac{1}{n}*P(2)$
    
    $P(n) = \frac{1}{n}(1 + P(n-1) + \dots + P(n-i) + \dots + P(2))$
    
    $P(2) = \frac{1}{2}$
\end{center}

Угадываем функцию: $P(n) = \frac{1}{2}$\\

\textbf{Задача 8} (5 баллов)

Пусть мужик производит эксперимент, который может завершиться любым из $N$ способов, причем $i$-й результат происходит (независимо от мужика) с вероятностью  $p_i$. Пусть мужик может врать или говорить правду вне зависимости от того, какой результат наблюдает (хотя его ответ, естественно, от наблюдения зависит), причем говорит правду с вероятностью  $p_{true}$, а врет с вероятностью $p_{lie} = 1 - p_{true}$. Если он говорит правду, он называет результат, который имеет место. Если он врет, то он равновероятно говорит любой из оставшихся $N-1$  вариантов. Требуется найти вероятность того, что произошло условие $i$, при условии, что мужик сказал, что произошло условие $i$.\\

A -- мужик, сказал что произошел i-й результат.

B --  произошел i-й результат.\\

$P(B\vert A) = \frac{P(A\vert B)*P(B)}{P(A)}$

$P(B) = p_i$ 

$P(A\vert B) = p_{true}$\\

$P(A) = P(A\vert B)*P(B) + P(A\vert \overline{B})*P(\overline{B}) = p_{true}*p_i + \frac{1 - p_{true}}{N-1}*(1 - p_i)$\\

Ответ: $P(B\vert A) = \frac{p_i*p_{true}}{p_{true}*p_i + \frac{1 - p_{true}}{N-1}*(1 - p_i)}$

\end{document}
