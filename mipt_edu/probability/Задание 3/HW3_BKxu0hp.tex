\documentclass{article}
\usepackage[utf8]{inputenc}
\usepackage[fleqn]{amsmath}
\usepackage{amssymb}
\usepackage{mathtools}

\usepackage[T1,T2A]{fontenc}
\usepackage[utf8]{inputenc}
\usepackage{natbib}
\usepackage{graphicx}
\usepackage[left=2cm,right=2cm,
    top=2cm,bottom=2cm,bindingoffset=0cm]{geometry}
\newcommand\tab[1][0.7cm]{\hspace*{#1}}
\title{Домашняя работа 3 (дедлайн -- 15:00 24.10.19)}

\begin{document}

\maketitle
\textbf{Задача 1} (5 баллов)
Пусть $\xi, \eta \sim$ Exp(1) -- независимые случайные величины. Найдите распределение случайной величины $\frac{\xi}{\xi + \eta}$ (подсказка: рассмотрите преобразование обратное к преобразованию $(\xi, \eta) \longrightarrow (\zeta,\theta), \space \zeta = \frac{\xi}{\xi + \eta}, \space \theta = \xi + \eta$ и выразите $f_{\zeta, \theta}(z, u)$, используя $f_{\zeta, \theta}(z, u)$).\\
\\
\textbf{Решение:} \\
Распределение Exp(1)\\
$f_{exp} = e^{-x}, x \geq 0$\\
$f_{exp} = 0, x \leq 0$\\
$F_{exp} = 1 - e^{-x}, x \geq 0$\\
$F_{exp} = 0, x \leq 0$\\
\\
$\zeta = \frac{\xi}{\xi + \eta}$\\
$\theta = \xi + \eta$\\
\\
Прямое преобразование:\\
$(\xi, \eta)$ с координатами $(x_1, x_2)$ $\mapsto$ $(\zeta, \Theta)$ с $(y_1, y_2)$\\
$y_1 = \frac{x_1}{x_1 + x_2}$\\
$y_2 = x_1 + x_2$\\
Обратное:\\
$x_1 = y_1*y_2$\\
$x_2 = y_2 + y_1*y_2$\\
Рассмотрим плотность при:\\
$x_1 > 0$\\
$x_2 > 0$\\
следовательно:
$0 < y_1 < 1$\\
$y_2 > 0$\\
в остальных случаях она равна нулю.\\
Якобиан обратного перехода:\\
$J = y_2 - y_1*y_2 + y_1*y_2 = y_2$\\
Плотность распределения:\\
$f_{\zeta,\theta}(y_1, y_2) = f_{\xi,\eta}(y_1*y_2, y_2 - y_1*y_2)*\vert J\vert$\\
$F_{\xi,\eta} = F_{exp}(x1)*F_{exp}(x_2) = 1 - e^{-x_1} - e^{-x_2} - e^{-x_1 - x_2}$\\
$f_{\xi,\eta} = \frac{\partial^2(e^{-x_1 - x_2})}{\partial x_1\partial x_2} = e^{-x_1 - x_2} = e^{-y_2}$\\
$f_{\zeta,\theta} = e^{-y_2}*y_2$\\
Получаем, что плотность не зависит от $y_1$, а значит:\\
$f_{\zeta} = 1$, при $0 < y_1 < 1$\\
$f_{\zeta} = 0$, в остальных случаях.\\

\textbf{Задача 2} (3 балла)
Пусть $\xi \sim \mathrm{Poly}(k, p_1, p_2, \dots, p_n)$. Покажите, что $\xi_i \sim \mathrm{Bi}(k, p_i)$
\\


\textbf{Решение:}\\
$$\xi = \frac{k!}{k_1!...k_n!}p_1^{k_1}...p_n^{k_n}$$\\
Надо доказать, что:\\
$$P(\xi_i = k_i) = C_k^{k_i}p_i^{k_i}(1-p_i)^{k-k_i}$$\\
$$P(\xi_i = k_i) = \sum\limits_{k_1,...,k_n\setminus k_i}P(\xi_1 = k_1, ..., \xi_n = k_n)$$\\
Доказательство сводится к равенству:\\
$$C_k^{k_i}p_i^{k_i}(1-p_i)^{k-k_i} = \sum\limits_{k_1,...,k_n\setminus k_i}\frac{k!}{k_1!...k_n!}p_1^{k_1}...p_n^{k_n}$$\\
$$\frac{1}{(n-n_i)!}(1-p_i)^{k-k_i} = \sum\limits_{k_1,...,k_n\setminus k_i}\frac{1}{k_1!...k_{i-1}!...k_{i+1}!...k_n!}p_1^{k_1}...p_{i-1}^{k_{i-1}}...p_{i+1}^{k_{i+1}}...p_n^{k_n}$$\\
$$(1-p_i)^{k-k_i} = \sum\limits_{k_1,...,k_n\setminus k_i}\frac{(n-n_i)!}{k_1!...k_{i-1}!...k_{i+1}!...k_n!}p_1^{k_1}...p_{i-1}^{k_{i-1}}...p_{i+1}^{k_{i+1}}...p_n^{k_n}$$\\
И справа и слева от знака равенства написана вероятность что фиксированные $(k - k_i)$ "шаров" не попадут в i-й "ящик" при полиномиальном распределении.\\
чтд\\

\textbf{Задача 3} (5 баллов)
Случайный вектор $\xi = (\xi_1, \xi_2)$ имеет равномерное распределение в треугольнике с вершинами в точках $(-1, 0), \space (0, 1), \space (1, 0)$. Найти распределение случайной величины $\eta =\frac{\xi_1 + \xi_2}{2}$ 
\\


\textbf{Решение:}\\
Прямое преобразование:\\
$\theta_1 = \frac{\xi_1-\xi_2}{2}$\\
$\theta_2=\eta$\\
Новые координаты $(y_1, y_2)$\\
Старые $(x_1, x_2)$\\
При переходе в новые координаты треугольник $(-1, 0), \space (0, 1), \space (1, 0)$ переходит в $(-0.5, -0.5), \space (0.5, 0.5), \space (0.5, -0.5)$\\

Так как якобиан обратного преобразования равен 2, новая плотность вероятности в треугольнике:\\
$f_{\theta_1,\theta_2}(y_1,y_2) = 2f_{\xi_1,\xi_2}(y_1+y_2,y_1-y_2)$\\
Проинтегрировав по $y_1$ от -inf до +inf (геометрически) получаем:\\
 \begin{equation*}
 f_{\eta}(y) = 
 \begin{cases}
   2y+1, & |y|\leq \frac{1}{2}\\
   0, & \text{иначе}
 \end{cases}
\end{equation*}

\textbf{Задача 4} (3 балла)
В каждую $i$-ую единицу времени живая клетка получает случайную дозу облучения $X_i$, причем $\{X_i\}_{i=1}^{t}$ имеют одинаковую функцию распределения $F_X(x)$ и независимы в совокупности для любого $t$. Получив интегральную дозу облучения, равную $\nu$, клетка погибает. Оценить среднее время жизни клетки $\mathrm{E} T$.
\\

\textbf{Решение:}\\
Расшарил решение задачи 51 из https://mccme.ru/ium/postscript/s12/gasnikov-tasks.pdf\\

\textbf{Задача 5} (4 балла)
Пусть $N$ -- случайная величина, принимающая натуральные значения, $\left\{ \xi_i 
\right\}_{i=1}^\infty$ -- некоррелированные одинаково распределенные случайные 
величины с конечными математическими ожиданиями и дисперсиями, не 
зависящие от $N$.
Рассмотрим $S_N = \sum\limits_{i=1}^N \xi_i$. Посчитайте $\mathrm{D} S_N$.
\\

\textbf{Решение:}\\
По тождеству Вальда:\\
$$E(S_N) = E(N)E(\xi_1)$$\\
\\
Из того, что св некоррелируемые следует:\\
$E\xi_i \xi_j = E\xi_i E\xi_j$\\
t -- число\\
$$
E(\sum\limits_{i=1}^{t}\xi_i)^2 = 
E(\sum_{i=1}^{t}\xi_i^2) + 2E\sum_{i\neq j}^{t}\xi_i\xi_j\ =
\sum_{i=1}^{t}E\xi_i^2\ + 2\sum_{i\neq j}^{t}E\xi_i E\xi_j = tE\xi_1^2 + t(t-1)(E\xi_1)^2 = t(t(E\xi_1)^2 + D\xi_1)
$$\\
Через условное матожидание:\\
$$
ES_N^2 = 
\sum\limits_{n=1}^{\infty}P(N=t)E(S_N^2|N = t) = 
\sum\limits_{n=1}^{\infty}P(N=t)E(S_t^2) = 
\sum\limits_{n=1}^{\infty}P(N=t)E(\sum\limits_{i=1}^{t}\xi_i)^2 = (E\xi_1)^2 EN^2 + EN D\xi_1
$$\\
$$D S_N = E(S_N^2)-(ES_N)^2$$\\
$$D S_N = D\xi_1 EN + (E\xi_1)^2 DN$$


\textbf{Задача 6} (5 баллов)
Пусть $\xi_1, \xi_2, \dots, \xi_n$ -- независимые одинаково распределённые с.в. с конечным мат.ожиданием, $\eta_n = \xi_1 + \xi_2 + \dots + \xi_n$. Доказать, что

$$\mathbb{E}(\xi_1 \vert \eta_n, \eta_{n+1}, \dots) = \frac{\eta_n}{n}$$

\textbf{Решение:}\\
$$E(\eta_n | \eta_n,\ \eta_{n+1},\ ...) = E(\xi_1+...+\xi_n|\ \eta_n,\ \eta_{n+1},\ ...) = \eta_n$$\\
Из о.р.\\
$$\eta_n = nE(\xi_1|\eta_n,\ \eta_{n+1},...)$$\\

\end{document}
