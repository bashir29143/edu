\documentclass[a4paper,12pt]{article}
\usepackage[T2A]{fontenc}
\usepackage[utf8]{inputenc}
\usepackage[english,russian]{babel}
\usepackage{amsmath,amsfonts,amssymb,amsthm,mathtools}

\begin{document}
Good morning everybody.\\

Let me start with the question: does unemployment increase crimerate?
Today I’m going to tell you about relation between unemployment and crimerate.\\

Because of the limited time, I’ll cover two areas: statistics and reasons of determined relation.\\

I’ll be happy to answer your questions at the end of my talk.\\

So, let’s look first at stastics. Criminality is one of the most worrying phenomena that affect economic development and social well-being.
According to historical data from the about 100 Italian provinces over the period 2000 to 2005, results are in line with the predictions of the economic model of crime. In this model unemployment have a positive correlation with all crime rates. It means that if unemployment increases crimerate increases too.\\

Let’s turn now to reasons of relation. Economic stress on unemployed parents leads to inadequate parenting practices, which in turn, increases the risk of teenager’s involvement in crimes. The research was conducted in Australia in order to analyze the relation between unemployment and crime. According to this research unemployment leads to the creation of huge income disparities in society, thus, in turn, leading to an increase in crime rate. Unemployment may lead to several factors, which may, in turn, force people to take the path of crime. For instance, unemployment may lead to social vices, such as poverty and malnutrition, which may make some people turn to crimes.\\

Let me summarise what we have looked at. The unemployment have a positive correlation with all crime rates.
That’s all I have to say.\\

At last, I hope I have succeeded in showing you that unemployment definitely has impact on crime rate.\\
\\
Thanks for your attention.\\
\\
Do you have any questions?\\

\end{document}